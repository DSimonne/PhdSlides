\begin{frame}{The material and pressure gap}

    \begin{columns}

        \column[T]{0.35\textwidth}
        
        \huge\textcolor{Important}{"A long standing conundrum in the catalysis community emerged at the interface between surface science and heterogeneous catalysis, better known as the pressure and materials gap."}
        
        \large\textcolor{Important}{Nature Catalysis editorial, 2018.}
        
        \column[T]{0.6\textwidth}

        Pressure gap:
        \begin{itemize}
            \item Typical industrial pressures between 1-12 bar and temperatures ranging from 1073-1223 K
            \item Most of the studies are performed at low pressures ($< 10^{-2} mbar$)
            \item Near-ambient pressure experiment
        \end{itemize}

        \vspace{2cm}
        Material gap:
        \begin{itemize}
            \item Platinum–rhodium wires, diameters in the range of 60–80 $\mu m$
            \item Pt nanoparticles and single crystals
        \end{itemize}

% Nowa- days, stacks of up to 50 knitted gauzes composed of platinum–rhodium wires with rhodium contents of 5–10 wt.% and wire diameters in the range of 60–80 μm have replaced the long term industry standard of woven platinum gauzes [2–4].

%cIn operation the catalyst gauzes undergo onsiderable reconstruction starting with smooth wires and ending with corroded wires covered with fractal structures in the shape of cauliflowers

    \end{columns}

\end{frame}